% new paragraph
The advancement of many engineering systems relies on novel design methodologies, design formulations, design representations, and other advancements. In this dissertation, we consider three broad design domains: architecture, plant, and control. These domains cover most of the potential design decision elements in an actively-controlled engineering system. In this dissertation, strategic aspects of this combined problem are addressed.

% new paragraph
The task of representing and generating candidate architectures is addressed with methods developed based on colored graphs built by enumerating all perfect matchings of a specified catalog of components. The proposed approach captures all architectures under specific assumptions. General combined plant and control design (or co-design) problems are examined. Previous work in co-design theory imposed restrictions on the type of problems that could be posed. Here many of those restrictions are lifted. The problem formulations and optimality conditions for both the simultaneous and nested solution strategies are given along with a detailed discussion of the two methods. Direct transcription is also discussed as it enables the solution of general co-design problems by approximating the problem. Motivated primarily by the need for efficient methods to solve certain control problems that emerge using the nested co-design method, an automated problem generation procedure is developed to support easy specification of linear-quadratic dynamic optimization problems using direct transcription and quadratic programming. Pseudospectral and single-step methods (including the zero-order hold) are all implemented in this unified framework and comparisons are made.

% new paragraph
Three detailed engineering design case studies are presented. The results from the enumeration and evaluation of all passive analog circuits with up to a certain number of components are used to synthesize low-pass filters and circuits that match a certain magnitude response. Advantages and limitations of enumerative approaches are highlighted in this case study, along with comparisons to circuits synthesized via evolutionary computation; many similarities are found in the topologies. The second case study tackles a complex co-design problem with the design of strain-actuated solar arrays for spacecraft precision pointing and jitter reduction. Nested co-design is utilized along with a linear-quadratic inner loop problem to obtain solutions efficiently. A simpler, scaled problem is analyzed to gain general insights into these results. This is accomplished with a unified theory of scaling in dynamic optimization. The final case study involves the design of active vehicle suspensions. All three design domains are considered in this problem. A class of architecture, plant, and control design problems which utilize linear physical elements is discussed. This problem class can be solved using the methods in this dissertation.