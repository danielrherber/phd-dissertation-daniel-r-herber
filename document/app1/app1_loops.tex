\section{Avoiding Loops\label{sec:app1:loops}}

Under specific NSCs, we can exclude loops during the graph generation process. A loop is an edge that connects a vertex to itself \cite[p.~25]{Diestel2000a}. 

\subsection{Theory}

Consider a component type that is both mandatory (\ref{ch2:s3}) and each edge is required to be unique (\ref{ch2:s7}).
Then a feasible graph cannot have any loops for this component type since the number of edges would not be unique.
We cannot make the same assumption if a component type needs to have each edge be unique but is not a mandatory component.
Consider a two-port, nonmandatory component type with one replicate.
Now, if we want a connected graph with all components except this specific two-port component, then a loop is required since each port must be filled. Therefore, this enhancement can be implemented with NAND logic where only a mandatory component type with each connection required to be unique is excluded from having loops.

This is not a general enhancement since it requires specific NSCs.

\subsection{Implementation}

\begin{vAlgorithm}[!ht]{\columnwidth}{0em}
\LinesNumbered
\DontPrintSemicolon

\SetKwFunction{find}{find}
\SetKwFunction{eye}{eye}
\SetKwFunction{length}{length}
\SetKwFunction{any}{any}

\SetKwInOut{Input}{Input}
\SetKwInOut{Output}{Output}

\caption{Limit potential connections based on loops.\label{alg:app1:loops}}

\Input{%
        \xvbox{2.5mm}{$\xvar{A}$} -- expanded potential adjacency matrix \\
        \xvbox{2.5mm}{$\xvar{M}$} -- vector indicating if a component replicate is mandatory \\
        \xvbox{2.5mm}{$\xvar{U}$} -- vector indicating if a component replicate requires unique connections \\
      }
\Output{%
        \xvbox{2.5mm}{$\xvar{A}$} -- expanded potential adjacency matrix
       }

\BlankLine

\If(\tcc*[f]{some loops should be excluded}){$\any(\xvar{M} \land \xvar{U})$}{

$\xvar{N}$ $\leftarrow$ $\length(\xvar{M})$ \tcc*{total number of component replicates}

$\xvar{iDiag}$ $\leftarrow$ $[1 : \xvar{N}+1 : \xvar{N}^2]$ \tcc*{indices for the diagonal elements}

$\xvar{A}(\xvar{iDiag})$ $\leftarrow$ $!\left(\xvar{M} \land \xvar{U}\right)$ \tcc*{assign NAND between M and U to the diagonal\label{alg:app1:loops-l4}}

}

\end{vAlgorithm}

Algorithm~\ref{alg:app1:loops} is the pseudocode for the implementation of this enhancement.
The implementation is centered around modifying the expanded potential adjacency matrix ($\xvar{A}$) before the graph generation algorithm is called.
The total number of components is found and then the linear index values for the diagonal of expanded potential adjacency matrix are computed.
Finally, in line~\ref{alg:app1:loops-l4}, NAND logical operator between $\xvar{M}$ and $\xvar{U}$ is assigned to the diagonal of $\xvar{A}.$

\subsection{Examples}

\subsubsection{Example 1\label{sec:app1:loops-ex1}}

The base three-tuple and NSCs for this example are specified as:
\begin{align}
C = \{ \xcolor{X}, \xcolor{Y} \}, \ \ R = [2\ 2], \ \ P = [2\ 2], \ \ \gls{S}_3 \text{ with } \gls{M} = [0\ 1], \ \ S_6 \text{ with } \gls{U} = [1 \ 1]
\end{align}

\noindent We will consider an all unity $\xvar{A}$ but this could be any expanded potential adjacency matrix. The second and third inputs to Alg.~\ref{alg:app1:loops} are:
\begin{align}
\xvar{M} = [0\ 0\ 1\ 1], \quad \xvar{U} = [1\ 1\ 1\ 1], \quad !\left(\xvar{M} \land \xvar{U}\right) = [1\ 1\ 0\ 0]
\end{align}

\noindent Now, Alg.~\ref{alg:app1:loops} modifies the expanded potential adjacency matrix as:
\begin{align}
\xvar{A} = \begin{blockarray}{ccccc}
& \xcolor{X} & \xcolor{X} & \xcolor{Y} & \xcolor{Y} \\
\begin{block}{c[cccc]}
\xcolor{X} & 1 & 1 & 1 & 1 \\
\xcolor{X} & 1 & 1 & 1 & 1 \\
\xcolor{Y} & 1 & 1 & 1 & 1 \\
\xcolor{Y} & 1 & 1 & 1 & 1 \\
\end{block}
\end{blockarray}
\xrightarrow{\xvar{A}(\xvar{iDiag}) \leftarrow !\left(\xvar{M} \land \xvar{U}\right)}
\xvar{A} = \begin{blockarray}{ccccc}
& \xcolor{X} & \xcolor{X} & \xcolor{Y} & \xcolor{Y} \\
\begin{block}{c[cccc]}
\xcolor{X} & \bm{1} & 1 & 1 & 1 \\
\xcolor{X} & 1 & \bm{1} & 1 & 1 \\
\xcolor{Y} & 1 & 1 & \bm{0} & 1 \\
\xcolor{Y} & 1 & 1 & 1 & \bm{0} \\
\end{block}
\end{blockarray}
\end{align}
\vspace{-2em}

\noindent where $\xcolor{Y}$ is mandatory and unique connections are required so corresponding diagonal entries in $\xvar{A}$ were zeroed. Table~\ref{tb:app1:loops-ex1} compares the original algorithm with the enhancement for this example. There is a reduction in candidate graphs generated while the number of unique graphs remains the same.

\begin{table}[!ht]
\centering
\caption{Comparison (loops, \nameref{sec:app1:loops-ex1}).\label{tb:app1:loops-ex1}}
\begin{tabular}{r | c | c | c}
\hline \hline
& Orig & En & Orig/En \\
\hline
Candidate Graphs & 26 & 16 & 1.63 \\ 
Unique Graphs & 3 & 3 & 1 \\
Generation Time (s) & 0.0024 & 0.0022 & 1.09 \\
Total Time (s) & 0.0086 & 0.0070 & 1.23 \\
\hline \hline
\end{tabular}
\end{table}

\subsubsection{Example 2\label{sec:app1:loops-ex2}}

The base three-tuple and NSCs for this example are specified as:
\begin{align}
C = \{ \xcolor{X}, \xcolor{Y} \}, \ \ R = [2\ 2], \ \ P = [2\ 2], \ \ S_3 \text{ with } M = [1\ 1], \ \ S_6 \text{ with } U = [1\ 1]
\end{align}

\noindent We will consider an all unity $\xvar{A}$ but this could be any expanded potential adjacency matrix. The second input to Alg.~\ref{alg:app1:loops} is:
\begin{align}
\xvar{M} = [1\ 1\ 1\ 1], \quad \xvar{U} = [1\ 1\ 1\ 1], \quad !\left(\xvar{M} \land \xvar{U}\right) = [0\ 0\ 0\ 0]
\end{align}

\noindent Now, Alg.~\ref{alg:app1:loops} modifies the expanded potential adjacency matrix as:
\begin{align}
\begin{aligned}
\xvar{A} = \begin{blockarray}{ccccc}
& \xcolor{X} & \xcolor{X} & \xcolor{Y} & \xcolor{Y} \\
\begin{block}{c[cccc]}
\xcolor{X} & 1 & 1 & 1 & 1 \\
\xcolor{X} & 1 & 1 & 1 & 1 \\
\xcolor{Y} & 1 & 1 & 1 & 1 \\
\xcolor{Y} & 1 & 1 & 1 & 1 \\
\end{block}
\end{blockarray}
\xrightarrow{\xvar{A}(\xvar{iDiag}) \leftarrow !\left(\xvar{M} \land \xvar{U}\right)}
\xvar{A} = \begin{blockarray}{ccccc}
& \xcolor{X} & \xcolor{X} & \xcolor{Y} & \xcolor{Y} \\
\begin{block}{c[cccc]}
\xcolor{X} & \bm{0} & 1 & 1 & 1 \\
\xcolor{X} & 1 & \bm{0} & 1 & 1 \\
\xcolor{Y} & 1 & 1 & \bm{0} & 1 \\
\xcolor{Y} & 1 & 1 & 1 & \bm{0} \\
\end{block}
\end{blockarray}
\end{aligned}
\end{align}
\vspace{-2em}

\noindent Since both component types are mandatory and unique connections are required, the entire diagonal is zeroed.
Table~\ref{tb:app1:loops-ex2} compares the original algorithm with the enhancement for this example. There is a reduction in candidate graphs generated while the number of unique graphs remains the same.

\begin{table}[!ht]
\centering
\caption{Comparison (loops, \nameref{sec:app1:loops-ex2}).\label{tb:app1:loops-ex2}}
\begin{tabular}{r | c | c | c}
\hline \hline
& Orig & En & Orig/En \\
\hline
Candidate Graphs & 26 & 11 & 2.36 \\ 
Unique Graphs & 2 & 2 & 1 \\
Generation Time (s) & 0.0024 & 0.0022 & 1.09 \\
Total Time (s) & 0.0086 & 0.0060 & 1.43 \\
\hline \hline
\end{tabular}
\end{table}