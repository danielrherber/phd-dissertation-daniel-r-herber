\section{Avoiding Multi-Edges\label{sec:app1:multiedges}}

Under specific NSCs, we can exclude multi-edges during the graph generation process. A multi-edge is two or more edges that are incident to the same two vertices \cite[p.~25]{Diestel2000a}.

\subsection{Theory}

Consider when each edge is required to be unique (\ref{ch2:s7}).
Due to the sequential nature of the tree algorithm, a single edge must be added between two components before a second edge is added; thus, creating a multi-edge.
Therefore, when the first edge is added between two components, we can utilize the expanded potential adjacency matrix to disallow any further connections between the components.
Since a feasible graph would not have any multi-edges when each edge is required to be unique, the tree algorithm with this enhancement will continue generating the desired graph structure space.

This enhancement should not be applied on loops since loops are occasionally needed to remove components (see Sec.~\ref{sec:app1:loops} for the handling of loops). Also, this is not a general enhancement since it requires specific NSCs.

\subsection{Implementation}

\begin{vAlgorithm}[!ht]{0.9\columnwidth}{0em}
\LinesNumbered
\DontPrintSemicolon

\SetKwFunction{find}{find}
\SetKwFunction{eye}{eye}
\SetKwFunction{length}{length}

\SetKwInOut{Input}{Input}
\SetKwInOut{Output}{Output}

\caption{Limit potential connections based on multi-edges.\label{alg:app1:multiedges}}

\Input{%
        \xvbox{2.5mm}{$\xvar{A}$} -- expanded potential adjacency matrix \\
        \xvbox{2.5mm}{$\xvar{U}$} -- vector indicating if a component replicate requires unique connections\\
        \xvbox{2.5mm}{$\xvar{iR}$} -- component index for right port \\
        \xvbox{2.5mm}{$\xvar{iL}$} -- component index for left port \\
      }
\Output{%
        \xvbox{2.5mm}{$\xvar{A}$} -- expanded potential adjacency matrix
       }

\BlankLine

\If(\tcc*[f]{either component requires unique connections}){$\xvar{U}(\xvar{iL}) \lor \xvar{U}(\xvar{iR})$}{
	\If(\tcc*[f]{don't do for self loops}){$\xvar{iR} \neq \xvar{iL}$}{

	$\xvar{A}(\xvar{iR}, \xvar{iL})$ $\leftarrow$ $0$ \tcc*{limit this connection\label{alg:app1:multiedges-l3}}

	$\xvar{A}(\xvar{iL}, \xvar{iR})$ $\leftarrow$ $0$ \tcc*{limit this connection\label{alg:app1:multiedges-l4}}

	}
	
}

\end{vAlgorithm}

Algorithm~\ref{alg:app1:multiedges} is the pseudocode for the implementation of this enhancement.
The implementation is centered around modifying the expanded potential adjacency matrix ($\xvar{A}$) when a new edge is created.
This enhancement is only called if either component replicate in the edge requires unique connections.
Additionally, this enhancement is only called if the edge is not a loop (see Sec.~\ref{sec:app1:loops} for the handling of loops).
If both of these conditions are met, then the corresponding entries in the expanded potential adjacency matrix are zeroed in lines~\ref{alg:app1:multiedges-l3} and \ref{alg:app1:multiedges-l4}.

This enhancement is inserted between lines~\ref{alg:app1:tree-A2} and \ref{alg:app1:tree-if} of Alg.~\ref{alg:app1:tree} using the local copy $\xvar{A2}$.

\subsection{Example}

The base three-tuple and NSCs for this example are specified as:
\begin{align}
C = \{ \xcolor{W}, \xcolor{X}, \xcolor{Y}, \xcolor{Z} \}, \quad R = [1\ 1\ 1\ 1], \quad P = [3\ 3\ 3\ 3], \quad S_6 \text{ with } U = [1\ 1\ 1\ 1]
\end{align}

\definecolor{forestgreen}{RGB}{0,155,85}
\definecolor{brickred}{RGB}{182,50,28}

\newcommand{\rhilght}[1]{\cancel{#1}}
\newcommand{\ghilght}[1]{#1}

\noindent Consider one path during the graph generation process when the  reduced potential adjacency matrix is initially all ones:
\begin{align*}
\begin{aligned}
& \underbrace{
\begin{blockarray}{ccccc}
& \xcolor{W} & \xcolor{X} & \xcolor{Y} & \xcolor{Z} \\
\begin{block}{c[cccc]}
\xcolor{W} & 1 & 1 & 1 & 1 \\
\xcolor{X} & 1 & 1 & 1 & 1 \\
\xcolor{Y} & 1 & 1 & 1 & 1 \\
\xcolor{Z} & 1 & 1 & 1 & 1 \\
\end{block}
\end{blockarray}
}_{\textstyle \text{Iter.~0, }\xvar{V} = [3\ 3\ 3\ 3] }
\rightarrow
\underbrace{
\begin{blockarray}{ccccc}
& \xcolor{W} & \xcolor{X} & \xcolor{Y} & \xcolor{Z} \\
\begin{block}{c[cccc]}
\xcolor{W} & 1 & \bm{0} & 1 & 1 \\
\xcolor{X} & \bm{0} & 1 & 1 & 1 \\
\xcolor{Y} & 1 & 1 & 1 & 1 \\
\xcolor{Z} & 1 & 1 & 1 & 1 \\
\end{block}
\end{blockarray}
}_{\textstyle \text{Iter.~1, }\xvar{V} = [\bar{2}\ \bar{\ghilght{2}}\ \ghilght{3}\ \ghilght{3}] }
\rightarrow
\underbrace{
\begin{blockarray}{ccccc}
& \xcolor{W} & \xcolor{X} & \xcolor{Y} & \xcolor{Z} \\
\begin{block}{c[cccc]}
\xcolor{W} & 1 & 0 & \bm{0} & 1 \\
\xcolor{X} & 0 & 1 & 1 & 1 \\
\xcolor{Y} & \bm{0} & 1 & 1 & 1 \\
\xcolor{Z} & 1 & 1 & 1 & 1 \\
\end{block}
\end{blockarray}
}_{\textstyle \text{Iter.~2, }\xvar{V} = [\bar{1}\ \rhilght{2}\ \bar{\ghilght{2}}\ \ghilght{3}] }
\rightarrow
\\
%
\rightarrow
& \underbrace{
\begin{blockarray}{ccccc}
& \xcolor{W} & \xcolor{X} & \xcolor{Y} & \xcolor{Z} \\
\begin{block}{c[cccc]}
\xcolor{W} & 1 & 0 & 0 & \bm{0} \\
\xcolor{X} & 0 & 1 & 1 & 1 \\
\xcolor{Y} & 0 & 1 & 1 & 1 \\
\xcolor{Z} & \bm{0} & 1 & 1 & 1 \\
\end{block}
\end{blockarray}
}_{\textstyle \text{Iter.~3, }\xvar{V} = [\bar{0}\ \rhilght{2}\ \rhilght{2}\ \bar{\ghilght{2}}] }
\rightarrow
\underbrace{
\begin{blockarray}{ccccc}
& \xcolor{W} & \xcolor{X} & \xcolor{Y} & \xcolor{Z} \\
\begin{block}{c[cccc]}
\xcolor{W} & 1 & 0 & 0 & 0 \\
\xcolor{X} & 0 & 1 & \bm{0} & 1 \\
\xcolor{Y} & 0 & \bm{0} & 1 & 1 \\
\xcolor{Z} & 0 & 1 & 1 & 1 \\
\end{block}
\end{blockarray}
}_{\textstyle \text{Iter.~4, }\xvar{V} = [0\ \bar{1}\ \bar{\ghilght{1}}\ \ghilght{2}] }
\rightarrow
\underbrace{
\begin{blockarray}{ccccc}
& \xcolor{W} & \xcolor{X} & \xcolor{Y} & \xcolor{Z} \\
\begin{block}{c[cccc]}
\xcolor{W} & 1 & 0 & 0 & 0 \\
\xcolor{X} & 0 & 1 & 0 & \bm{0} \\
\xcolor{Y} & 0 & 0 & 1 & 1 \\
\xcolor{Z} & 0 & \bm{0} & 1 & 1 \\
\end{block}
\end{blockarray}
}_{\textstyle \text{Iter.~5, }\xvar{V} = [0\ \bar{0}\ \rhilght{1}\ \bar{\ghilght{1}}] }
\end{aligned}
\end{align*}

\noindent where the matrix above represents $\xvar{A}$, $\bar{\Box}$ indicates this component was selected for an edge, and $\cancel{\Box}$ indicates the connection was disallowed. Each iteration added a pair of zeros to the potential adjacency matrix.

Table~\ref{tb:app1:multiedge-ex1} compares the original algorithm with the enhancement for this example.
There is a reduction in candidate graphs generated while the number of unique graphs remains the same.

\begin{table}[!ht]
\centering
\caption{Comparison (multi-edge).\label{tb:app1:multiedge-ex1}}
\begin{tabular}{r | c | c | c}
\hline \hline
& Orig & En & Orig/En \\
\hline
Candidate Graphs & 211 & 46 & 4.59 \\ 
Unique Graphs & 1 & 1 & 1 \\
Generation Time (s) & 0.0086 & 0.0037 & 2.32 \\
Total Time (s) & 0.015 & 0.0070 & 2.14 \\
\hline \hline
\end{tabular}
\end{table}