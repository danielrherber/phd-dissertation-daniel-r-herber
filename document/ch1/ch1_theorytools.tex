%----------------------------------------------------------
\section{Solution Generation Challenges\label{sec:ch1:theorytools}}

There are a number of challenges associated with the design freedom found in combined architecture, plant, and control problems.
The complexity of individual problems can be quite significant, sometimes even rendering the problem intractable, unless this complexity is handled appropriately.  
Efficiently automating the key tasks in the solution generation process is essential to arriving at desirable solutions in a practical manner.
Here we highlight three important generation tasks: candidate architecture generation, model generation, and optimization problem generation.

%----------------------------------------------------------
\subsection{(Automated) Candidate Architecture Generation\label{sec:ch1:archgen}}

Exploring different architectures requires an appropriate conceptual framework that allows for modifications to the appropriate elements in the architecture.
A straightforward, commonly used representation is an adjacency matrix where the nonzero entries in the matrix represent connections between elements in the architecture.
Candidate architectures could be a different set of nonzero values in the adjacency matrix.
However, not all architecture representations are equally useful.
Some might produce many infeasible systems or too many candidates.
Others might produce many architectures with poor performance, or are not amendable to some optimization procedure.
An example alternative representation framework was developed for electrical circuits where a sequence of low-level instructions are used to generate a circuit \cite{Lohn1999a}.
These instructions iteratively add new elements to the circuit in  topologically different ways.
A manual alternative would include experts proposing candidate architectures, which may be appropriate in some cases, but often may not support comprehensive design space exploration.
In many traditional engineering design problems, the architecture is fixed so this abstract representation is not typically needed.

%----------------------------------------------------------
\subsection{(Automated) Model Generation\label{sec:ch1:modelgen}}

Given some architecture specification (e.g.,~a graph), we need to create a suitable model (some representation of the architecture that can predict performance and identify if constraints are satisfied) for use in the optimization problems described in Sec.~\ref{sec:ch1:process}.
Certain modeling methodologies support this task such as bond graph
modeling \cite{Borutzky2010a} or block diagram-based modeling \cite{matlab-simulink}. 
Other techniques have been developed for specific types of problems such as solid isotropic material with penalization for structural optimization \cite{Bendsoe2004a} and \glsfirst{MNA} for electrical circuits \cite{Ho1975a}.
Nevertheless, for many design problems, some investment will be needed to generate models efficiently and in an automated manner.
In traditional engineering design problems, the model is fixed for the given architecture, but can vary based on the plant/control design variables.

%----------------------------------------------------------
\subsection{(Automated) Optimization Problem Generation}

For different candidate architectures of the same design problem,  the optimization problem that predicts its performance may vary.
If the model is different, there may be different plant and control variables.
Different constraints may be present depending on the components in the architecture (e.g.,~we only need control actuator bounds if the component is present).
Since all problem elements (e.g.,~number and types of variables, objective function form, constraints, model) could change between architectures, forming and solving the optimization problem automatically is important for solution efficiency.
Understanding the general optimization problem structure for any candidate architecture is key so that it can be leveraged to find solutions faster and more robustly.
For some types of optimization problems, such as ones with infinite-dimensional constraints and variables, additional work is need to obtain (approximate) solutions.
Frequently, only a single optimization problem form needs to be posed and solved in traditional engineering design problems.