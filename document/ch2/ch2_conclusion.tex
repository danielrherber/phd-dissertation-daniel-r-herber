%------------------
\section{Summary}

Architecture design is a challenging problem and this chapter presents some theory for generating candidate architectures with perfect matchings.
A \mypm{} approach is a graph numerical representation scheme that completely covers the design space that is needed in many architecture design problems.
It ensures certain frequency and degree requirements are met on specific list of potential components.
Enumeration of architecture design spaces can provide coverage and insights not currently possible with generative design approaches since enumeration approaches allow the designer to make specifications that they understand such as constraints and potential components, rather than rules for how things should be connected.

% new paragraph
A number of general network structure constraints are fully outlined with the specifics of checking their satisfaction with available graph analysis tools.
The colored graph isomorphism problem is discussed in great detail including the distinction between port-type and component-type isomorphisms.
The limited number of full isomorphism checks and the efficiency of Alg.~\ref{alg:ch2:cip} demonstrate that larger than expected architecture design spaces can be obtained.
A basic and improved tree search algorithm that avoids port-type isomorphisms was shown and is a primary example of how constraints can be naturally satisfied without reducing the design space.

% new paragraph
The various case studies are initial insights into the true nature of the class of architecture problems studied in this dissertation.
Consider again that there are only 12 unique graphs in Fig.~\ref{fig:ch2:unique3} of 2,097,152 adjacency matrices and 135,135 \mypm{}s.
For the suspension case study, a wide variety of network structure constraints were included based on natural requirements such as no direct connection between the sprung and unsprung masses.
Other constraints were added to avoid duplicate dynamic models such as the avoidance of parallel cycles and series connections between the 2-port components.
Moreover, constraints based on the experience and intuition of the designer were also included which limited the design space in a predictable manner such as the requirement that no parallel components can be connected together.

% new paragraph
Future graph generation algorithms can use these insights to suitably address the unique challenges associated with architecture design problems.
A number of directions are possible with the \mypm{} framework including deeper analysis of the structural properties of \mypm{} graphs, reduction of the number of graphs generated by the tree search algorithm, and the development of structured sampling approaches that result in nearly all unique graphs.