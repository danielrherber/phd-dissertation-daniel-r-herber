%------------------
\section{Discussion}

It is clear in the case studies that the number of unique designs is much smaller than the upper bounds given by either permutations of the adjacency matrix or a \mypm{} approach. We also can directly visualize the effect of adding specific NSCs. NSCs limited the architecture design space, but in a predictable manner. One such example was not allowing parallel components to be connected together in \nameref{sec:ch2:example4}. A 5-port parallel connection was no longer possible (i.e.,~a 3-port and 4-port parallel components connected), but this NSC greatly reduced the number of graphs generated excluding many infeasible graphs, such as ones where parallel connection paths existed between a connected $\xcolor{S}$ or $\xcolor{U}$. Therefore, the addition of this NSC was a decision based on the tradeoff between  coverage of the architecture design space and efficiency.
\nameref{sec:ch2:example4} also demonstrated that fairly large problem sizes can be enumerated with the improved tree search algorithm provided enough constraints are present. Also, all possible subgraphs that are connected and complete appear in the set of unique designs without any NSCs (e.g.,~all graphs in Fig.~\ref{fig:ch2:unique2} appear as subgraphs in Fig.~\ref{fig:ch2:unique1}).

All reported unique solutions have a corresponding \mypm{} number. This number may not be unique since other \mypm{} numbers maybe isomorphic to the resulting $G^{CC}$. We see an example of two different \mypm{}s producing isomorphic graphs (\mypm{}~462 in Fig.~\ref{fig:ch2:model1_4} and \mypm{}~678 in Fig.~\ref{fig:ch2:unique1:g}).
While checking for isomorphisms can be computationally demanding, there typically is only a small subset of graphs that need the full isomorphism check as many comparisons fail with the simple checks and filters. Algorithm~\ref{alg:ch2:cip} can be useful to any architecture design problem no matter how the set of colored graphs is generated.
Many of the results and algorithms assumed simple components, but structured components (i.e.,~a planetary gear) can be readily included. Replacing a simple component type with an equivalent structured component type would simply have the effect of increasing the number of unique designs.

The previous sections only considered enumeration constructing a specific graph structure space. However, many problems are too large for the proposed enumeration algorithms because computational limits (e.g.,~memory needed to store the graphs and computation time of Alg.~\ref{alg:ch2:cip}) and evaluation limits (i.e., too many graphs are generated and we cannot evaluate all of them). Therefore, we need to consider methods that provide suitable \textit{exploration} of the desired design space.

Both the pure \mypm{} approach and the tree search algorithms have nice properties such as the high likelihood of producing feasible, nonisomorphic graphs while not limiting the design space. A stochastic sampling of the unique integers between 1 and $\mathcal{D}(N_P)$ can produce any arbitrary number of \mypm{} graphs. However, more structured sampling approaches may be preferred. Consider the unique graphs in Fig.~\ref{fig:ch2:unique1}. We could have tested all \mypm{}s between 227 to 913 and found all unique graphs. A \mypm{} approach does exhibit some interesting similarly-preserving properties (e.g.,~the graphs for a given \mypm{} number and the next integer value have a high likelihood of containing similar edges). Further exploration of the structural properties of \mypm{} graphs could lead to better sampling techniques that still cover the desired design space.

We can further consider ways to structure the exploration space with the coupon collector's problem. This problem, stated in a form relevant to this article, is: 
\begin{quote}
There are $n$ unique graphs and at each trial, a new graph is chosen at random (with replacement). Let $m$ be the number of trials. What is the expected number of trials such that all $n$ unique graphs are selected?
\end{quote}

\noindent The expected number of trials needed grows as $O(n\ln(n))$ \cite{Flajolet1992a}. Some of the assumptions in the problem are not directly satisfied such replacement and the probability distribution of choosing a particular unique graph but further study on the structure of $\mathcal{G}_1$ may yield exploration that `collects' most of the unique graphs in a more efficient manner.

The tree search algorithms may also be used for exploration. On Line~\ref{alg:ch2:cip:5} of Alg.~\ref{alg:ch2:cip}, we can randomly select an edge to add from $\xvar{I}$ instead of trying all possible edges. Therefore, the tree can be explored stochastically. Since the number of branches from a leaf varies, the probability of arriving at a certain final edge set is not equal (these probabilities can be calculated by assuming the tree is a Markov chain). Since the tree search algorithms cover the same desired graph structure space as the pure \mypm{} approach, can we selectively sample the tree and have some predictions on when all unique designs are found? These questions are left as future work items. 

Finally, it is important to describe the specific uses of the proposed algorithms. They are suitable for problems that are represented by undirected colored graphs under the component/port paradigm \cite{Mittal1989a, Snavely1993a, Munzer2013a}.
Enumeration is appropriate for certain problem classes (primarily determined by size). It may also be appropriate for searching for all possible enhancing structures \cite{Munzer2013a}. Enumeration has been useful for generating lists of organic molecules \cite{Carhart1975a, Faulon2003b, Ruddigkeit2012a}, finding all geometries of electrical circuits \cite{Foster1932a}, identifying all biological network architectures that achieve specific behaviors \cite{Ma2009a}, enumerating different gear trains \cite{Pennestri2015a, Castillo2002a}, and determining all hybrid powertrain configurations for a set list of components \cite{Bayrak2016a}.

Exploration is suitable for sampling the design space for much larger problems \cite{Wyatt2014a}. These samples could be used as visualizations for expert evaluation or starting points for generative approaches. The unrestricted graphs from a \mypm{} approach could also be used in conjunction with feature extraction algorithms to develop generative rules that are not based solely on experience and intuition (where the features are subgraphs that provide desired benefits to the architecture) \cite{Berlingerio2009a}.