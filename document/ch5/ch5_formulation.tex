%-------------------------------------------------------------------
\section{Linear-Quadratic Dynamic Optimization} \label{sec:ch5:lqdo}

In this section, we begin by describing the general NLDO formulation and then a \qp{} is defined. Then, with an assumed property of the solution method, we will characterize the \lqdo{} formulation that supports solution via {\qp}s.

%-------------------------------------------------------------------
\subsection{General Nonlinear Dynamic Optimization} \label{sec:ch5:gnldo}

% Brief statement of general nonlinear optimal control
Dynamic-system design permits the optimization of: control trajectories, $\gls{olc}(t)$; static parameters, $\gls{parameters}$; and the time horizon defined by the boundary values $\gls{time}_{\glsfoo[noindex]{initial}}$ and $t_{\glsfoo[noindex]{final}}$. Written as an infinite-dimensional mathematical program, the NLDO formulation is:%
\begin{subequations}%
\label{eq:ch5:NLDO}
\begin{align}
\min_{\bm{u}(t),\bm{p},t_0,t_f} \qquad& \gls{objective} = \int_{t_0}^{t_f} \mathcal{L} \big(t,\bm{\xi}(t),\bm{u}(t), \bm{p} \big) dt + \mathcal{M}\big(\bm{p}, t_0, \bm{\xi}(t_0), t_f, \bm{\xi}(t_f) \big) \label{eq:ch5:NLobj} \\
\text{subject to:} \qquad &  \hspace{0.0in} 
\dot{\bm{\xi}}(t) - \glsname{f}\big(t, \bm{\xi}(t), \bm{u}(t), \bm{p} \big) = \bm{0} \label{eq:ch5:NLdyn} \\
& \gls{equality}\big(t, \bm{\xi}(t), \bm{u}(t), \bm{p}, t_0, \bm{\xi}(t_0), t_f, \bm{\xi}(t_f) \big) = \bm{0}  \label{eq:ch5:NLeq} \\
& \gls{inequality}\big(t, \bm{\xi}(t), \bm{u}(t), \bm{p}, t_0, \bm{\xi}(t_0), t_f, \bm{\xi}(t_f) \big) \leq \bm{0}  \label{eq:ch5:NLineq}
\end{align}
\end{subequations}%

\noindent where Eqn.~(\ref{eq:ch5:NLobj}) defines the objective function with Lagrange $\gls{lagrange}$ and Mayer $\gls{mayer}$ terms.
Equation~(\ref{eq:ch5:NLdyn}) enforces the first-order \glsfoo[noindex]{ODE} that describes the dynamic behavior of the states $\gls{states}(t)$.
Equation~(\ref{eq:ch5:NLeq}) enforces the algebraic equality constraints, and Eqn.~(\ref{eq:ch5:NLineq}) enforces the algebraic inequality constraints.
Most NLDO problems only contain certain elements of this general formulation.
Many presentations of general NLDO reorganize Eqns.~(\ref{eq:ch5:NLeq})--(\ref{eq:ch5:NLineq}) by partitioning the constraints into those which depend on time-varying quantities and those which do not.
The time-varying constraints are termed \textit{path} constraints, and time-independent constraints termed \textit{boundary} constraints (as was done in Chapters~\ref{ch:3} and \ref{ch:4}) \cite{Betts2010a, Biegler2010a, Herber2014a}.
Some formulations also include the states as optimization variables (primarily motivated by the eventual solution method and the states are still constrained by Eqn.~(\ref{eq:ch5:NLdyn})).

\subsection{Quadratic Program}

In this chapter, we are concerned with a subclass of Prob.~(\ref{eq:ch5:NLDO}) that can be solved numerically as a finite-dimensional \qp. A \qp{} is defined as:%
\begin{subequations}%
\label{eq:ch5:basicQP}
\begin{align}
\min_{\gls{xqp}} \qquad& \frac{1}{2}\mathbf{X}^{\glsfirst{transpose}} \gls{hqp} \mathbf{X} + \gls{fqp}\tran \mathbf{X} + \gls{cqp} \label{eq:ch5:qpH} \\
\text{subject to:} \qquad& \gls{Aqp}_{\glsfirst{noteq}} \mathbf{X} = \gls{Bqp}_{e} \label{eq:ch5:Ae} \\
& \mathbf{A}_{\glsfirst{notineq}} \mathbf{X} \leq \mathbf{B}_{i} \label{eq:ch5:Ai} \\
& \myunderbar{\mathbf{X}\gls{myunderbar}} \leq \mathbf{X} \leq \myoverbar{\mathbf{X}\gls{myoverbar}} \label{eq:ch5:qpbounds}
\end{align}
\end{subequations}%

\noindent where $\mathbf{X}$ is the set of optimization variables, and all other terms in the formulation are real-valued with no dependence on $\mathbf{X}$.
$\mathbf{H}$ is known as the Hessian matrix and is symmetric.
We note that the \qp{} formulation in Prob.~(\ref{eq:ch5:basicQP}) is decidedly more structured than NLDO formulation in Prob.~(\ref{eq:ch5:NLDO}).
The optimal solution to a quadratic program exists and is the global optimum if the objective is a convex quadratic function and the feasible set of Prob.~(\ref{eq:ch5:basicQP}) is nonempty \cite{Pang1983a}.
There are a variety of efficient algorithms for solving {\qp}s, including  active-set \cite{Pang1983a}, interior-point \cite{Altman1999a}, and augmented Lagrangian \cite{Delbos2005a} methods.

\subsection{Linear-Quadratic Dynamic Optimization Problem Formulation} \label{sec:ch5:lqdoform}

In this section, a subclass of Prob.~(\ref{eq:ch5:NLDO}) is defined that, when combined with a solution method, can generate a \qp{} in the form of Prob.~(\ref{eq:ch5:basicQP}) that approximates the infinite-dimensional solution. 
There are two aspects that determine if a particular infinite-dimensional problem can be formulated as \qp: 1) the structure of the objective function and constraints in Prob.~(\ref{eq:ch5:NLDO}), and 2) the solution method that approximates the infinite-dimensional problem as a finite one.
To address the latter aspect, we will only consider order-maintaining methods.
An order-maintaining method, for example, would have the following property: if a term is linear in the infinite-dimensional problem, then it will be approximated as a set of variables that appear linearly in the finite-dimensional problem.
An equivalent property will need to be true for quadratic terms.

First, we denote the vector of optimization variables as:
\begin{align}\label{eq:ch5:x:cont}
\gls{x} = \begin{bmatrix} \bm{u} \\ \bm{\xi} \\ \bm{p} \end{bmatrix}
\end{align}

\noindent Both elements of the time horizon are missing because they do not permit {\qp}s (see Sec.~\ref{sec:ch5:mtp}).
In addition, the states are included to support proper analysis of the problem class and the eventual solution methods.
We also see $\bm{\xi}(t_0)$ and $\bm{\xi}(t_f)$ directly in the formulation, so we define an expanded set of optimization variables as:
\begin{align} \label{eq:ch5:varindices}
\gls{expanded} = \begin{lbmatrix}{1} \bm{x}_c \\ \bm{x}_d \end{lbmatrix} \ \text{where:} \ \ 
\bm{x}_c = \begin{lbmatrix}{1} \bm{u} \\ \bm{\xi} \end{lbmatrix}
:= \begin{lbmatrix}{1} \bm{x}_1 \\ \bm{x}_2 \end{lbmatrix} \  \text{and} \ \ 
\bm{x}_d = \begin{lbmatrix}{1} \bm{p} \\ \bm{\xi}(t_0) \\ \bm{\xi}(t_f) \end{lbmatrix}
:= \begin{lbmatrix}{1} \bm{x}_3 \\ \bm{x}_4 \\ \bm{x}_5 \end{lbmatrix}
\end{align}

\noindent where $\bm{x}_{\glsfirst{continuous}}$ and $\bm{x}_{\glsfirst{discrete}}$ are collections of the continuous (infinite-dimensional) and discrete (finite-dimensional) optimization variables, and indexed variables $\bm{x}_i$ are equivalent to the variable in the same row in the vector directly to the left in the same equation (e.g.,~$\bm{u} := \bm{x}_1$). 

%-------------------------------------------------------------------
\subsubsection{Linear Dynamics}

A good place to start is with the dynamics expressed in Eqn.~(\ref{eq:ch5:NLdyn}) as it is a unique feature of DO. An appropriate choice for the state dynamics is a linear nonhomogeneous differential equation:
\begin{align} \label{eq:ch5:fdyn}
\bm{f} = \bm{f}^{\glsfirst{notqp}} := \gls{dynA}(t)\bm{\xi}(t) + \gls{dynB}(t) \bm{u}(t) + \gls{dynG}(t)\bm{p} + \gls{dynd}(t) 
\end{align}

\noindent where $\{\bm{A},\bm{B},\bm{G}\}$ are the \glsfirst{LTV} state/input/parameter matrices and $\bm{d}(t)$ is a disturbance.

Many authors consider different variations on Eqn.~(\ref{eq:ch5:fdyn}).
The simplest form is that of a \glsfirst{LTI} system: $\bm{f} = \bm{A}\bm{\xi} + \bm{B} \bm{u}$ \cite{Lack1967a, Sala2007a}.
The next most common form is the LTV system without $\bm{G}$ or $\bm{d}$ \cite{Wang1992a}.
The addition of the disturbance does appear in a number of formulations \cite{Hampton1996a, Popescu2009a, Li2015a, Bryson1975a, Han2012a}.
No general formulations are seen which include $\bm{p}$, but this may be due to the limited number of problems where $\bm{p}$ shows up linearly in $\bm{f}$. 
However, this demonstrates that even the simplest mixed parameter-control problems are not \lqdo{} problems \cite{Herber2014a}.
A more general form is $\gls{dynE}(t)\dot{\bm{\xi}} = \bm{f}^{QP}$, known as descriptor form \cite{Gerdts2015a, Campbell2013a}.
Here we will only consider that case where $\bm{E}$ is nonsingular so it may be written as $\dot{\bm{\xi}} = \bm{E}^{-1}\bm{f}^{QP}$. 

One may be tempted to integrate this differential equation and would arrive at the following:
\begin{align} \label{eq:ch5:statesol}
\bm{\xi}(t) = \bm{\Phi}(t,t_0) \bm{\xi}(t_0) + \int_{t_0}^{t}  \bm{\Phi}(t,\tau) \big( \bm{B}(\tau) \bm{u}(\tau) + \bm{G}(\tau) \bm{p} + \bm{d}(\tau) \big) d\tau
\end{align}

\noindent where $\gls{transition}$ is the state-transition matrix which describes the dynamics of the homogeneous system (i.e.,~the integral in Eqn.~(\ref{eq:ch5:statesol}) is zero) \cite{Chen1999a}.
The most general transition matrix is given by the Peano-Baker series.
However, if $\bm{A}$ is time-invariant, then the state-transition matrix is $e^{\bm{A}(t-\tau)}$.
Some solution methods will utilize these properties, but in general, the differential equation is challenging to utilize directly.

%-------------------------------------------------------------------
\subsubsection{Quadratic Objective Function}

The \qp{} objective function in Eqn.~(\ref{eq:ch5:qpH}) is a polynomial with degree two; therefore, we will choose a general quadratic cost functional containing appropriate elements of $\tilde{\bm{x}}$:%
\allowdisplaybreaks[1]%
\begin{gather} \label{eq:ch5:QOFLagrange}
\mathcal{L} = \mathcal{L}^{QP} := 
\underbracket{\sum_{i=1}^5 \sum_{j=1}^5 \bm{x}_i\tran  \gls{Lbm}_{ij}(t) \bm{x}_j}_{\textstyle H^{\mathcal{L}} = \tilde{\bm{x}}\tran  \bL \tilde{\bm{x}} }
\ + \
\underbracket{\sum_{j=1}^5 \gls{lbm}_j\tran (t) \bm{x}_j}_{\textstyle F^{\mathcal{L}} = \bl\tran   \tilde{\bm{x}}}
\ + \
c^{\mathcal{L}}(t) \\ \label{eq:ch5:QOFMayer}
\mathcal{M} = \mathcal{M}^{QP} := 
\underbracket{\sum_{i=3}^5 \sum_{j=3}^5 \bm{x}_i\tran \gls{Mbm}_{ij} \bm{x}_j}_{\textstyle H^{\mathcal{M}} = {\bm{x}}\tran_d \bM {\bm{x}}_d }
\ + \
\underbracket{\sum_{j=3}^5 \gls{mbm}\tran_j \bm{x}_j}_{\textstyle F^{\mathcal{M}} = \bmm\tran  \bm{x}_d}
\ + \
c^{\mathcal{M}}
\end{gather}
\allowdisplaybreaks[0]%

\noindent where we require that $\bL_{ij} = \bL_{ji}\tran$ and $\bM_{ij} = \bM_{ji}\tran$ so that both are symmetric since $\mathbf{H}$ needs to be symmetric.
We note that the Mayer term only includes the discrete optimization variables because including any element of $\bm{x}_c$ would not result in a scalar quantity\footnote{We can also note that since the time horizon is fixed in \lqdo, $\mathcal{M}$ is redundant since $\bm{M}_{ij}/(t_f-t_0)$ could be added to $\bm{L}_{ij}$ for the appropriate indices.
To facilitate more natural descriptions, all terms are considered.}.

The Lagrange and Mayer terms in Eqns.~(\ref{eq:ch5:QOFLagrange})--(\ref{eq:ch5:QOFMayer}) include all possible time-varying quadratic and linear terms in the most general expression.
While most of the applications utilize a few select time-variant and time-invariant terms for their performance index \cite{Hampton1996a, Popescu2009a, Campbell2013a, Gerdts2015a, Tang2007a, Jerez2011a, Bashein1971a}, there are a few applications in the literature which include all the time-varying $\bm{x}_c$ dependent terms in the linear and quadratic terms, but only includes $\bm{\xi}(t_f)$ terms in linear and quadratic Mayer terms \cite{Wang1992a, Sideris2010a, Han2012a}.
Constants in Lagrange and Mayer terms ($c^{\mathcal{L}}$ and $c^{\mathcal{M}}$) are not widely used, but $c^{\mathcal{M}}$ term appears in a few references \cite{Gerdts2015a}.
We also note that if there are no quadratic terms (i.e.,~$\bm{L}=\bm{0}$ and $\bm{M}=\bm{0}$), then the \lqdo{} problem can be solved with linear programming (a special case of quadratic programming) \cite{Boyd2009a}.

%-------------------------------------------------------------------
\subsubsection{Additional Linear Constraints} 

In most DO problems there are additional constraints that need to be enforced.
In the context of \lqdo, these can all be expressed as the following general linear equality and inequality constraint forms:%
\allowdisplaybreaks[1]%
\begin{gather}
h = h^{QP} := \sum_{j=1}^5 \gls{Ybm}_{j}(t)\tran \bm{x}_j - \hat{Y}(t) = \bm{Y}\tran \tilde{\bm{x}} - \hat{Y} = 0 \label{eq:ch5:hqp} \\
g = g^{QP} := \sum_{j=1}^5 \gls{Zbm}_{j}(t)\tran \bm{x}_j - \hat{Z}(t) = \bm{Z}\tran \tilde{\bm{x}} - \hat{Z} \leq 0 \label{eq:ch5:gqp}
\end{gather}%
\allowdisplaybreaks[0]%

\noindent This representation of the constraints mixes the path and boundary definitions mentioned in Sec.~\ref{sec:ch5:gnldo}. In Sec.~\ref{sec:ch5:alg:lin:constraints} it will be shown that a straightforward procedure exists to determine which class (path or boundary) is most appropriate for a specified constraint. 

As with the previous \lqdo{} elements, there are a number of common constraint types seen in literature that are captured by these general linear constraints.
There is a broad class of boundary condition constraints that only contain the initial and final states \cite{Gerdts2015a} including prescribed initial conditions \cite{Lack1967a, Bryson1975a, Wang1992a, Anderson2007a, Tang2007a, Sideris2010a, Han2012a}, prescribed final or terminal conditions \cite{Lack1967a, Bashein1971a, Bryson1975a, Jerez2011a}, and periodic conditions \cite{Herber2014a}.
Initial and final state conditions need not be only equality constraints \cite{Jerez2011a}. Mixed control-state constraints are also possible \cite{Sideris2010a, Bryson1975a, Gerdts2015a, Jerez2011a, Han2012a}.
This type of constraint commonly includes only $\bm{x}_c$ terms, but the whole collection $\tilde{\bm{x}}$ is possible and necessary such as when a parameter is used to represent maximum or absolute values (see Sec.~\ref{sec:ch5:absolute:values} and Sec.~\ref{sec:ch5:minmax:objective}).

There is a special subclass of Eqn.~(\ref{eq:ch5:gqp}) which contains only single linear term:%
\begin{subequations}%
\label{eq:ch5:simplebounds}
\begin{align}
\myunderbar{\bm{x}}(t) - \bm{x}_j \leq \bm{0} \\
\bm{x}_j - \overbar{\bm{x}}(t)  \leq \bm{0}
\end{align}
\end{subequations}
\noindent This subclass is mentioned because it fits the form of Eqn.~(\ref{eq:ch5:qpbounds}) more so than either Eqn.~(\ref{eq:ch5:Ae})--(\ref{eq:ch5:Ai}).
If $\myunderbar{\bm{x}}$ and $\overbar{\bm{x}}$ are not time-varying (see Ref.~\cite{Lack1967a} where they can be time-varying), these constraints are commonly called saturation constraints or simple bounds \cite{Bashein1971a}.

%-------------------------------------------------------------------
\subsubsection{Comparison to Similar Formulations}

With all elements of the \lqdo{} problem, we will briefly compare with some other common formulations found in the literature.
Perhaps the most ubiquitous \lqdo{} problem is the finite-horizon \glsfoo[noindex]{LQR} problem \cite{Liberzon2012a}:%
\begin{subequations}%
\label{eq:ch5:lqr}
\begin{align}
\min_{\bm{u}(t)} \qquad& \int_{t_0}^{t_f} \left( \bm{\xi}\tran(t)\bm{L}_{22}\bm{\xi}(t)  + \bm{u}\tran(t)\bm{L}_{11}\bm{u}(t) \right) dt +  \bm{\xi}\tran(t_f)\bm{M}_{55} \bm{\xi}(t_f) \\
\text{subject to:} \qquad &  \hspace{0.0in} 
\dot{\bm{\xi}}(t) - \big(  \bm{A}\bm{\xi}(t) + \bm{B} \bm{u}(t) \big) = \bm{0} \\
& \bm{\xi}(t_0) - \bm{\xi}_0 = \bm{0}
\end{align}
\end{subequations}%

\noindent We note that the problem is time-invariant with no path constraints or mixed terms.
This problem structure is still amenable to deriving a set of reasonably simple optimality conditions in the form of a boundary value problem \cite{Bryson1975a, Liberzon2012a}.
The optimality conditions for different variations on Prob.~(\ref{eq:ch5:lqr}) have also been studied by Bryson and Ho~\cite{Bryson1975a}. These variations include an LTV system with $\bm{\xi}\tran \bm{L}_{21} \bm{u}$ terms, exactly zero terminal error (i.e.,~$\bm{\xi}(t_f) = \bm{0}$), and the nonhomogeneous equation $\bm{f} = \bm{A}(t) \bm{\xi}(t) + \bm{B}(t) \bm{u}(t) + \bm{d}(t)$.
Since these problem structures do not have path constraints, it is fairly straightforward to derive the optimality conditions.

% new paragraph
Many \glsfirst{MPC} paradigms utilize Prob.~(\ref{eq:ch5:lqr}) \cite{Borrelli2015a}. % [p.~167
The following additional constraints are also common in MPC formulations:
\begin{align}
x(t) \in \mathcal{X}, \quad u(t) \in \mathcal{U}, \quad \forall t \geq 0
\end{align}

\noindent where the sets $\mathcal{X}$ and $\mathcal{U}$ are polyhedra.
These sets can be represented by sets of linear constraints of form Eqns.~(\ref{eq:ch5:hqp})--(\ref{eq:ch5:gqp}), or even as simple bounds (see Sec.~\ref{sec:ch5:polyhedra}).
These polyhedra constraints are possible for more general forms of linear constraints, such as mixed input and state constraints \cite{Borrelli2015a}.

% new paragraph
An \lqdo{} formulation with many similar elements is studied by Sideris and Rodriguez~\cite{Sideris2010a} with the notation slightly amended to be consistent with this work:%
\begin{subequations}%
\label{eq:ch5:Sideris}
\begin{align}
\min_{\bm{u}(t)} \quad& \scalebox{0.94}{$\displaystyle  \int_{t_0}^{t_f} \left( \sum_{i=1}^2 \sum_{j=1}^2 \bm{x}_i\tran \bL_{ij}(t) \bm{x}_j + \sum_{j=1}^2 \bl_j\tran(t) \bm{x}_j \right) dt + \bm{\xi}\tran(t_f)\bm{M}_{55} \bm{\xi}(t_f) +  \bm{\xi}\tran(t_f)\bm{m}_5$} \\
\text{subject to:} \quad &  \hspace{0.0in} 
\dot{\bm{\xi}}(t) - \big(  \bm{A}(t) \bm{\xi}(t) + \bm{B}(t) \bm{u}(t) \big) = \bm{0} \\
& \bm{\xi}(t_0) - \bm{\xi}_0 = \bm{0} \\
& h = \sum_{j=1}^2 \bm{Y}_{j}(t)\tran \bm{x}_j - \hat{Y}(t) = 0
\end{align}
\end{subequations}%

\noindent Here we see an LTV system, an objective function with mixed and linear terms, and mixed linear equality conditions.
However, there are no inequality constraints nor general boundary conditions. 
Han et al.~\cite{Han2012a} include inequality constraints and the disturbance, along with most of the problem elements in Prob.~(\ref{eq:ch5:Sideris}).
From these select examples, we can see a diversity of formulations in the literature, and all fit with the given \lqdo{} framework.