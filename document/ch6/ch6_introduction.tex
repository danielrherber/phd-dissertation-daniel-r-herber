\section{Introduction\label{sec:ch6:introduction}}

% new paragraph
Synthesis of analog electric circuits is a complex and resource-intensive task.
Circuit-level synthesis involves two major attributes: 1) the topology (i.e.,~what components are present and how they are connected or the circuit structure) and 2) sizing (i.e.,~the selection of the component values).
Both attributes are required for a fully-defined circuit.
While there exist many computer-aided tools for performing network analysis and simulation \cite{Grimbleby1995a} as well as circuit sizing \cite{Das2007a}, there is still a need for high-quality synthesis methods that perform well on a variety of circuit synthesis problems.

% new paragraph
Any synthesis method must navigate the immense potential design space present in circuit synthesis design problems \cite{Mitea2011a}.
Often this vast design space leads to the reliance on experts and domain-specific knowledge \cite{Gan2010a}.
Alternatively, formal design methodologies have been developed for specific problem classes (e.g.,~linear frequency-domain filters) \cite{Grimbleby2000a}.
While these human-focused and specific formal approaches can be suitable for some applications, they often fail at finding desirable solutions for unfamiliar and complex design problems.
To address some of these issues, there has been considerable research to try to leverage existing design knowledge by formally incorporating heuristics \cite{Sussman1975a} and knowledge bases \cite{Harjani1989a} into various tools.

% new paragraph
More recent efforts have moved toward methods with minimal initial design knowledge \cite{Das2007a,  Grimbleby1995a, Gan2010a, Das2008a,  Lohn1999a}.
Methods that adopt this principle can better generate truly novel topologies and nonexperts can more easily parse the required inputs for the automation tool \cite{Das2007a, Grimbleby1995a, Goh2001a, Grimbleby2000a, Lohn1999a}.
Both evolutionary computation \cite{Das2007a, Das2008a, Gan2010a, Grimbleby1995a, Goh2001a, Grimbleby2000a, Koza1997a, Lohn1999a} and simulated annealing \cite{Ochotta1996a} are concepts that have been utilized to generate and select fully-defined circuits with minimal knowledge.
Evolutionary-based methods, in particular, have received significant attention due to their relative success at balancing between minimal initial knowledge, design space freedom, and computational expense \cite{Das2007a}.

% new paragraph
In evolutionary computation, an initial population (set of candidate solutions) is generated and iteratively updated through metaheuristics and stochastic operators \cite{Eiben2003a}.
While these approaches  can produce `semi-optimal' solutions \cite{Das2007a}, they still have some drawbacks. 
First, the underlying algorithms have a number of parameters that must be selected (e.g.,~crossover probability and population size) \cite{Eiben2003a, Grimbleby1995a, Goh2001a, Koza1997a, Lohn1999a}.
Often values are given, but the sensitivity of these parameters is not thoroughly investigated, or better yet, optimal recommendations.
Secondly, a poor initial population can result in delayed convergence \cite{Das2007a}.
Since these approaches are inherently stochastic, robustness is an important issue.
Even with a favorable initial population, using standard metaheuristics can be inefficient (although some recent work has started to address this issue with custom genetic algorithms \cite{Das2007a}).
It has also been observed that overly complex solutions (i.e.,~too many components) are often selected \cite{Goh2001a, Grimbleby1995a, Grimbleby2000a}. 
In general, a target response is more easily satisfied with a complex circuit than a simple one \cite{Grimbleby1995a}.
Finally, the global solution is not guaranteed, and it is hard to determine how far we are from the global optimum.
Addressing these shortcomings could lead to a truly efficient and effective synthesis tool.

% new paragraph
All of the previously mentioned synthesis methods have aimed to navigate the vast design space by selectively sampling from the complete design space.
An infrequently utilized approach is to actually test all topologies. %  (for good reason). 
In an enumerative synthesis approach, we generate and evaluate a complete listing of all possible circuit topologies under certain specifications.
Then we can simply select the best one with confidence in its optimality.
The obvious reason to not test all topologies is the rate at which such a complete listing grows with the number of candidate system components. Here we make a case for the proposed enumeration-based synthesis methodology to solve certain types of synthesis problems, as well as to generate knowledge that could aid more general and scalable solution of synthesis problems.

% new paragraph
Enumeration has been utilized previously for electrical circuit design.
Macmahon, among others, have considered the enumeration of series-parallel networks \cite{Macmahon1994a, Lomnicki1972a, Isokawa2016a}.
Foster provided the enumeration of geometrical circuits classified according to their nullity and rank \cite{Foster1932a}.
Bruccoleri et al.~systematically generated all the wide-band amplifier circuits with two MOS transistors \cite{Bruccoleri2001a, Klumperink1997a}. 
Enumeration has also been useful in other synthesis problems such as hybrid powertrains \cite{Bayrak2016a}, gear trains \cite{Castillo2002a}, and biological networks \cite{Ma2009a}.

% new paragraph
Before the enumeration-based methodology is described, it is important to visit the main motivations for this work, including the questions:
\begin{itemize}
\item Under what specifications/complexity is enumeration still feasible?

\item Have previous methods found the ``best'' circuit? How close was the found circuit to the ``best'' circuit?

\item What are the properties of the complete circuit structure space (e.g.,~how many unique circuits are there with up to a certain number of components)?

\item Are stochastically sampled (rather than enumerative) topologies generated with the proposed approach more useful for initial populations due to their structural properties?

\item Can a complete listing of circuits (both good and bad) be used to adapt an evolutionary approach to better address the synthesis task, perhaps through online learning or new metaheuristics?

\end{itemize}

\noindent These points will be addressed in Sec.~\ref{sec:ch6:discussion}.

% new paragraph
The remainder of the chapter is organized as follows. Section~\ref{sec:ch6:enumeration} describes the proposed enumeration-based synthesis methodology. Section~\ref{sec:ch6:examples} provides some examples for both frequency response and low-pass filter synthesis problems. Section~\ref{sec:ch6:discussion} is a discussion of the results and methodology. Section~\ref{sec:ch6:conclusion} provides the conclusions.