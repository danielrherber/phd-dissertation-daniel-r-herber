\section{Summary\label{sec:ch8:conclusion}}

The results in this case study demonstrate that changes to the vehicle suspension architecture can result in improved performance.
The purpose of these early-stage studies is to identify new architectures that could be investigated in the same level of detail that the few canonical architectures have received.
This case study utilized a newly developed paradigm for combined architecture, plant, and control design that can be applied to systems with linear physical elements.

% new paragraph
It remains future work to evaluate the entire set of possible 13,727 unique suspension architectures from Prob.~(\ref{eq:ch8:CRP}) \cite{Herber2017a}.
In addition, there are a number of improvements that can be made to the problem formulation.
Multiple road inputs should be considered simultaneously to give a better representation of all the environments the suspension will need to function in.
Frequency domain properties, such as suspension quality spectral density and control energy spectral density, could also be utilized for a more effective problem formulation  \cite{Fathy2003a}.